We now present the structure of the protocol using notation for key material,
elliptic curve operations and
identities mostly adopted from NIST SP 800-56A Rev. 3~\cite{sp800-56a-rev3}.
%
One notable difference is that we refer to the two roles executing the protocol
as the initiator $I$ and the responder $R$.
%
We do this to avoid confusing roles with the parties taking them on.
%
Values in the analysis are subscripted with $I$ and $R$ when necessary to
distinguish which role is associated with the values.
%

%------------------------------------------------------------------------- sub
\subsection{Preliminaries}
\label{sec:preliminaries}
Private/public key pairs are written
$\langle d_{t,\mathit{id}},\,Q_{t,\mathit{id}}\rangle$,
where $d$ is the private key, $Q$ the public key, $t \in \{e, s\}$ denotes
whether the key is ephemeral or static, and $\mathit{id}$ is the role or
party controlling the key pair.
%
When irrelevant, we drop the subscript or parts thereof.
%
Ephemeral key pairs are generated fresh for each instantiation of the protocol
and static key pairs are long-term keys used for authentication.
%
Static key pairs are suitable for either regular signatures or
challenge-response signatures.
%
When a party uses regular signatures for authentication, we say that they use
the \emph{signature based authentication method}, or \mSig{} for short.
%
When a party uses challenge-response signatures for authentication, we say that
they use the \emph{static key authentication method}, or \mStat{} for short.
%
The latter naming may appear confusing since signature keys are equally static,
but is chosen to make the connection to the \mSpec{} clear.
%
We adopt the challenge-response terminology for this style of authentication
from~\cite{DBLP:conf/crypto/Krawczyk05}.
%

\mEdhoc{} relies on \mCose{}~\cite{rfc8152} for elliptic curve operations and
transforming points into bitstrings, and we therefore abstract those as
follows.
%
Signatures and verification thereof using party $A$'s key pair are
denoted \mSign{A}$(\cdot)$ and \mVf{A}$(\cdot)$ respectively.
%
The DH-primitive combining a private key $d$ and a point $P$ is denoted
$\mDH(d,P)$.
%
We abuse notation and let these function symbols denote operations on both
points and the corresponding bitstrings.
%

%------------------------------------------------------------------------- sub
\subsection{Framework Structure}
\label{sec:framework}
\mEdhoc{}'s goal is to establish an \mOscore{} security context,
including session key material \mbox{denoted \mSessKey{}}, and optionally transfer
application data \mADone{}, \mADtwo{} and \mADthree{}.
%
To accomplish this, the
\mSpec{}~\cite{our-analysis-selander-lake-edhoc-00} gives
a three-message protocol pattern, shown in Figure~\ref{fig:edhocFramework}.
%
We first describe this pattern and the parts that are common to all key
establishment methods.
%
Then we describe authentication and derivation of keys in more detail.
%
The latter is what differentiate the key establishment methods from each other.
%

\subsubsection{Protocol Pattern}
The first two messages negotiate authentication methods \mMethod{} and a
ciphersuite \mSuites{}.
%
In \mMethod{}, the initiator $I$ proposes which authentication method each
party should use.
%
These may differ, leading to four possible combinations:
\mSigSig{}, \mSigStat{}, \mStatSig{} and \mStatStat{}.
%
We refer to these \emph{combinations} of authentication methods simply as
\emph{methods} to align with the \mSpec{} terminology.
%
The first authentication method in a combination is the one proposed for the
initiator and the last is the one proposed for the responder.
%
The responder $R$ may reject the choice of method
or cipher suite with an error message, resulting in negotiation across
multiple \mEdhoc{} sessions.
%
Our analysis excludes error messages.
%

\mEdhoc{}'s first two messages also exchange connection identifiers \mCi{} and
\mCr{}, and public ephemeral keys, \mGx{} and \mGy{}.
%
The connection identifiers \mCi{} and \mCr{}, described in Section 3.1 of the
\mSpec{}, deserve some elaboration.
%
The \mSpec{} describes these identifiers as not serving a security purpose for
\mEdhoc{}, but only as aiding message routing to the correct \mEdhoc{} processing
entity at a party.
%
Despite this, the \mSpec{} states that they may be used by \mOscore{}, or other
protocols using the established security context, without restricting how they
are to be used.
%
Because \mEdhoc{} may need them in clear-text for routing, \mOscore{} cannot
rely on them being secret.
%
Section 7.1.1 of the \mSpec{} requires the identifiers to be unique.
%
Uniqueness is defined to mean that $\mCi{} \not = \mCr{}$ for a given session
and the \mSpec{} requires parties to verify that this is the case.
%
The same section also require that \mOscore{} must be able to retrieve the
security context based on these identifiers.
%
The intended usage of \mCi{} and \mCr{} by \mOscore{} is not made specific and
therefore it is not clear which properties should be verified.
%
We verify that the parties agree on the established values.
%

The two last messages provide identification and authentication.
%
Parties exchange identifiers for their long-term keys, \mIdcredi{} and \mIdcredr{},
as well as information elements, \mAuthi{} and \mAuthr{}, to authenticate
that the parties control the corresponding long-term keys.
%
The content of \mAuthi{} and \mAuthr{} depends on the authentication method
associated with the corresponding long-term key.
%
For example, if $\mMethod{} = \mSigStat{}$, the responder $R$ must either
reject the offer or provide an \mIdcredr{} corresponding to a key pair
$\langle\mPriv{R},\ \mPub{R}\rangle$ suitable for use with challenge-response
signatures, and
compute \mAuthr{} based on the static key authentication method \mStat{}.
%
In turn, the initiator $I$ must provide an \mIdcredi{} corresponding to a key
pair $\langle\mPriv{I},\ \mPub{I}\rangle$ suitable for a regular signature,
and compute
\mAuthi{} based on the signature based authentication method \mSig{}.
%

\subsubsection{Authentication}
Regardless of whether \mStat{} or \mSig{} is used to compute \mAuthr{}, a
MAC is first computed over \mIdcredr{}, \mCredr{}, a transcript hash of the
information exchanged so far, and \mADtwo{} if included.
%
The MAC is the result of encrypting the empty string with the Authenticated
Encryption with Additional Data (AEAD) algorithm from the ciphersuite
\mSuites{}, using the mentioned information as additional data.
%
The MAC key is derived from the ephemeral key material
\mGx{}, \mGy{}, \mX{} and \mY{}, where $I$
computes $\mDH(\mX,\ \mGy)$ and $R$ computes $\mDH(\mY,\ \mGx)$, both resulting in
the same output in the usual DH way.
%

In case $R$ uses the \mSig{} authentication method, \mAuthr{} is $R$'s
signature over the MAC itself and the data that the MAC already covers.
%
In case $R$ uses the \mStat{} authentication method, \mAuthr{} is simply the
MAC itself.
%
However, when using \mStat{}, the key for the MAC is derived, not only from the
ephemeral key material, but also from $R$'s long-term key
$\langle\mPriv{R},\ \mPub{R}\rangle$.
%
For those familiar with \mOptls, this corresponds to the 1-RTT semi-static
pattern computing the MAC key \textsf{sfk} for the \textsf{sfin}
message~\cite{DBLP:conf/eurosp/KrawczykW16}.
%
The content of \mAuthi{} is computed in the corresponding way for the initiator
$I$.
%
In Figure~\ref{fig:edhocFramework}, we denote a MAC using a key derived from
both $\langle\mPriv{R},\ \mPub{R}\rangle$ and $\langle\mX,\ \mGx\rangle$ by
$\mathit{MAC}_I$, and a MAC using a key derived from
both $\langle\mPriv{I},\ \mPub{I}\rangle$ and $\langle\mY,\ \mGy\rangle$ by
$\mathit{MAC}_R$.
%

Parts of the last two messages are encrypted and integrity protected, as
indicated in~\ref{fig:edhocFramework}.
%
The second message is encrypted by XORing the output of the key derivation
function HKDF (see Section~\ref{sec:keysched}) on to the plain text.
%
The third message is encrypted and integrity protected by the AEAD algorithm
determined by the ciphersuite \mSuites{}.
%

\begin{figure}
\centering
\tikzset{>=latex, every msg/.style={draw=thick}, every node/.style={fill=none,text=black}}
\begin{tikzpicture}
    \node (ini) at (0, 0) {Initiator};
    \draw [very thick] (0, -0.25) -- (0,-2.3);
    \draw [very thick] (5.75, -0.25) -- (5.75,-2.3);
    \node (res) at (5.75,0) {Responder};
    \msg{1em}{ini}{res}{\mMsgone: \mMethod, \mSuites, \mGx, \mCi, \mADone};
    \msg{3em}{res}{ini}{\mMsgtwo: \mCi, \mGy, \mCr, \{\mIdcredr, \mAuthr, \mADtwo\}};
    \msg{5em}{ini}{res}{\mMsgthree: \mCr, \{\mIdcredi, \mAuthi, \mADthree\}};
    \draw [line width=1mm] (-0.75,-2.3) -- (0.75,-2.3);
    \draw [line width=1mm] (5.75-0.75,-2.3) -- (5.75+0.75,-2.3);
    \node (padding) at (0,-2.5) {};
    \end{tikzpicture}
    \begin{tabular}{|c|c|c|}
        \hline
        \mMethod & \mAuthi & \mAuthr\\
        \hline
        \mSigSig{} & $\mSign{I}(\cdot)$ & $\mSign{R}(\cdot)$ \\
        \mSigStat{} & $\mSign{I}(\cdot)$ & $\textit{MAC}_R(\cdot)$\\
        \mStatSig{} & $\textit{MAC}_I(\cdot)$ & $\mSign{R}(\cdot)$\\
        \mStatStat{} & $\textit{MAC}_I(\cdot)$ & $\textit{MAC}_R(\cdot)$\\
        \hline
    \end{tabular}
    \caption{Structure of \mEdhoc{}: $\{t\}$ means $t$ is encrypted and integrity
protected.}
\label{fig:edhocFramework}
\end{figure}

\subsubsection{Key Schedule}
\label{sec:keysched}
At the heart of \mEdhoc{} is the key schedule depicted in
Figure~\ref{fig:kdfdiagram}.
%
\mEdhoc{} uses two functions from the \mHkdf{} interface~\cite{rfc5869} to derive keys.
%
\mHkdfExtract{} 
constructs uniformly distributed key material from random input and a salt,
while \mHkdfExpand{} generates keys from key material and a salt.
%

The key schedule is rooted in the ephemeral DH key
\mGxy{}, which is computed as $\mDH(\mX, \mGy)$ by $I$ and as $\mDH(\mY, \mGx)$
by $R$.
%
From \mGxy{}, three intermediate keys \mPRKtwo, \mPRKthree{} and
\mPRKfour{} are derived during the course of protocol execution.
%
Each of them is used for a specific message in the protocol, and from
these intermediate keys, encryption and integrity keys
(\mKtwoe, \mKtwom{}, \mKthreeae, and \mKthreem) for that message are derived.
%
The salt for generating \mPRKtwo{} is the empty string.
%

The protocol uses a running transcript hash $th$, which includes the information
transmitted so far.
%
The value of the hash, denoted $th_i$ for the $i$th message, is included in key
derivations as shown in Figure~\ref{fig:kdfdiagram}.
%

Successful protocol execution establishes the session key material \mSessKey{}
for \mOscore{}.
%
\mSessKey{} can be considered a set that always includes \mGxy{}.
%
If the initiator uses the \mStat{} authentication method, \mSessKey{} also
includes
$\mDH(\mY,\ \mGi{}) = \mDH(\mPriv{I},\ \mGy)$, which we denote $P_I$.
%
If the responder uses the \mStat{} authentication method, it also includes
$\mDH(\mX,\ \mGr{}) = \mDH(\mPriv{R},\ \mGx)$, which we denote $P_R$.
%
From the session key material, a key exporter (\mEdhoc-Exporter) based on
\mHkdf{} is used to extract keys required for \mOscore{}.
%

\begin{figure*}[!h]
\centering
\scalebox{.785}{
% !TEX root =  protocol.tex

\begin{tikzpicture}[%
    >=latex,              % Nice arrows; your taste may be different
    start chain=going below,    % General flow is top-to-bottom
    node distance=4mm and 60mm, % Global setup of box spacing
    every join/.style={norm},   % Default linetype for connecting boxes
    ]
% ------------------------------------------------- 
% A few box styles 
% <on chain> *and* <on grid> reduce the need for manual relative
% positioning of nodes
\tikzset{
terminput/.style={rounded corners},
term/.style={rounded corners},
  base/.style={draw, thick, on chain, on grid, align=center, minimum height=4ex},
  dhkbox/.style={draw=cbgreen, fill=cbgreen!25, rectangle},
  dhk/.style={base, dhkbox},
  prkbox/.style={draw=cborange, fill=cborange!25, rectangle},
  prk/.style={base, prkbox},
  %hkdfext/.style={base, draw=Green3, fill=Green3!25, isosceles triangle, isosceles triangle apex angle=60, anchor=base, shape border rotate=-90, text width=6em},
  hkdfext/.style={base, draw=black, fill=none, rectangle},
  %hkdfexp/.style={base, draw=orange, fill=orange!50, isosceles triangle, isosceles triangle apex angle=60, anchor=base, shape border rotate=-90, text width=6em},
  hkdfexp/.style={base, draw=black, fill=none, rectangle},
  keybbox/.style={draw=cbnavy, fill=cbnavy!25, rectangle},
  keyb/.style={base, keybbox, text width=4em},
  % -------------------------------------------------
  norm/.style={->, draw, black},
  cond/.style={base, draw=black, dashed, fill=none, rectangle},
  txt/.style={base, draw=none, fill=none}
  }
% -------------------------------------------------
% Start by placing the nodes
%\node [terminput] (u1) {Salt};
%\node [hkdfext, join] (h1) {\mHkdfExtract};
\node [prk, join] (p2) {\mPRKtwo};
\node [cond, join] (c1) {R uses \mStat?};

\node [prk, below=6mm of c1.south] (p3) {\mPRKthree};
\draw [->, norm] (c1.south) -- (p3.north) node[midway, right] {N};

\node [cond, join, below=10mm of p3.south] (c2) {I uses \mStat?};
\node [prk, below=8mm of c2.south] (p4) {\mPRKfour};
\draw [->, norm] (c2.south) -- (p4.north) node[midway, right] {N};

\node [hkdfext, right=3cm of p3] (h3) {\mHkdfExtract};
\node [hkdfext, right=3cm of p4] (h5) {\mHkdfExtract};

\node [hkdfexp, shape border rotate=180, left= 2.5cm of p4] (h6) {\mHkdfExpand};
\node [keyb, join, left=3cm of h6] (k3) {\mKthreem};
\node [hkdfexp, shape border rotate=180, below= 0.8cm of h6] (h9) {\mHkdfExpand};
\node [txt, join, left=1cm of h9.west] (t4) {EDHOC-Exporter};

\node [hkdfexp, shape border rotate=180, left= 2.5cm of p3] (h4) {\mHkdfExpand};
\node [keyb, join, left=3cm of h4] (k2) {\mKtwom};

\node [hkdfexp, shape border rotate=180, left= 2.5cm of p2] (h2) {\mHkdfExpand};
\node [keyb, join, left=3cm of h2] (k1) {\mKtwoe};

\node [hkdfexp, shape border rotate=180, below= 0.8cm of h4] (h8) {\mHkdfExpand};
\node [keyb, below=0.8cm of k2] (k2b) {\mKthreeae};

\node [txt, left=1cm of k1.west] (t1) {Enc (XOR) \\ in m2};
\node [txt, left=1cm of k2.west] (t2) {\mMactwo~(signed if \\ R uses \mSig)};
\node [txt, left=1cm of k2b.west] (t2b) {\mAead\ in m3};
\node [txt, left=1cm of k3.west] (t3) {\mMacthree~(signed if \\ I uses \mSig)};

\draw [->, norm] (k1.west) -- (t1.east);
\draw [->, norm] (k2.west) -- (t2.east);
\draw [->, norm] (k2b.west) -- (t2b.east);
\draw [->, norm] (k3.west) -- (t3.east);

\draw [->, norm] (p3.south) ++(0,-0.5) -- (h8);
\draw [->, norm] (h8) -- (k2b);
\draw [->, norm] (p2) -- (h2); 
\draw [->, norm] (c1.east) -- ++(1.85,0) -- (h3.north) node[midway,above left] {Y};
%\draw [->, norm] (h3.south) -- ++(0,-1) -- ++(-3,0);
\draw [->, norm] (h3.west) -- (p3.east);
\draw [->, norm] (p3) -- (h4); 
\draw [->, norm] (c2.east) -- ++(1.91,0) -- (h5.north) node[midway,above left] {Y};
%\draw [->, norm] (h5.south) -- ++(0,-1) -- ++(-3,0);
\draw [->, norm] (h5.west) -- (p4.east);
\draw [->, norm] (p4) -- (h6);
\draw [->, norm] (p4.west) ++(-0.25,-0) -- ++(0,-0.8) -- (h9.east);

\node [hkdfext, right=3cm of p2] (h1) {\mHkdfExtract};
\node [dhk, right=2.7cm of h1] (p0) {$\mGxy$};
\node [terminput, text width=2em, below = 0.2cm of p0] (u1) {Salt};
\draw [->] (h1.west) -- (p2.east);
\draw [->] (u1.west) -- ++(-1.52,0) -- (h1.south);
\draw [->] (p0.west) -- (h1.east);


\node [dhk, right = 2.7cm of h3] (u2) {$P_R$};
\draw [->, norm] (u2.west) -- (h3.east);

\node [dhk, right = 2.7cm of h5] (u3) {$P_I$};
\draw [->, norm] (u3.west) -- (h5.east);


\node [term, above = 0.55cm of h4] (u5) {\mTHtwo};
\draw [->, dotted, shorten >=1mm] (u5) -- (h4);
\draw [->, dotted, shorten >=1mm] (u5) -- (h2);

\node [term, above = 0.5cm of h6] (u6) {\mTHthree};
\draw [->, dotted, shorten >=1mm] (u6) -- (h6);
\draw [->, dotted, shorten >=1mm] (u6) -- (h8);

%\node [term, below= 1cm of h9] (u7) {\mTHfour};
\node [term, below=0.4cm of p4] (u7) {\mTHfour};
%\draw [->, dotted, shorten >=1mm] (u7) -- (h9.south);
\draw [-> , dotted ] (u7.west) -- ([yshift=-0.4em] h9.east);

%\matrix [draw, ultra thick, double, below=2em of t1.east] {
%  \node [dhkbox, semithick, label=right:DH key] {}; \\
%  \node [prkbox, semithick, label=right:Intermediate key material] {}; \\
%  \node [keybbox, semithick, label=right:Key for \mAead{} or \mXor] {}; \\
%};

%
% ------------------------------------------------- 
% 
%\path (h2.east) to node [near start, yshift=1em] {$n$} (c3); 
%  \draw [o->,lccong] (h2.east) -- (p8);
%\path (p3.east) to node [yshift=-1em] {$k \leq 0$} (c4r); 
%  \draw [o->,lcnorm] (p3.east) -- (p9);
% -------------------------------------------------
% A last flourish which breaks all the rules
%\draw [->,MediumPurple4, dotted, thick, shorten >=1mm]
%  (p9.south) -- ++(5mm,-3mm)  -- ++(27mm,0) 
%  |- node [black, near end, yshift=0.75em, it]
%    {(When message + resources available)} (p0);
% -------------------------------------------------
\end{tikzpicture}


}
\caption{Key schedule: Light blue boxes hold DH keys ($P_e, P_I, P_R$),
orange boxes intermediate key material (\mPRKtwo, \mPRKthree, \mPRKfour), and
dark blue boxes keys for \mAead{} or \mXor{} encryption
(\mKtwoe, \mKtwom, \mKthreeae, \mKthreem).
Dashed boxes are conditionals.}
\label{fig:kdfdiagram}
\end{figure*}

As an illustrative example of the entire process, we refer to
Figure~\ref{fig:edhocsigstat}, which depicts the protocol pattern, operations
and key derivations for the \mSigStat{} method in more detail.
%
\begin{figure*}[ht]
\centering
\scalebox{.7}{
\tikzset{>=latex, every msg/.style={draw=thick}, every node/.style={fill=none,text=black}}
\begin{tikzpicture}
    \node (ini) at (0, 0) {Initiator};
    \draw [very thick] (0, -0.5) -- (0,-14.8);
    \draw [very thick] (9, -0.5) -- (9,-14.8);
    \node[below=0.5em of ini,fill=white] {$
    \begin{array}{c}
        \text{Knows}\ \langle\mPriv{I},\ \mPub{I}\rangle,\ \mIdcredi,\ \mIdcredr,\ \mADone,\ \mADthree
    \end{array}
    $};
    \node (res) at (9,0) {Responder};
    \node[below=0.5em of res,fill=white] {$
    \begin{array}{c}
        \text{Knows}\ \langle\mPriv{R},\ \mPub{R}\rangle, \ \mIdcredr,\ \mADtwo
    \end{array}$};
    \action{3.5em}{ini}{Generates $\mMethod,\ \mSuites,\ \mCi,\ \langle\mX{},\ \mGx\rangle$};
    \msg{6.5em}{ini}{res}{\mMsgone: \mMethod, \mSuites, \mGx, \mCi, \mADone};
    \action{7.0em}{res}{$
      \begin{array}{c}
          \text{Generates } \mCr,\ \langle\mY{},\ \mGy\rangle\\
          \ \ P_e = \mDH(\mY,\ \mGx{})\\
          \ \ P_R = \mDH(\mPriv{R},\ \mGx{})\\
        \mTHtwo = \mHash(\mMsgone, \langle \mCi, \mGy, \mCr \rangle)\\
        \mPRKtwo = \mHkdfExtract(\textrm{``\phantom{}''}, P_e) \\
        \mPRKthree = \mHkdfExtract(\mPRKtwo, P_R) \\
        \mKtwom = \mHkdfExpand(\mPRKthree, \mTHtwo) \\
        \mMactwo = \mAead(\mKtwom; \langle \mIdcredr, \mTHtwo, \mCredr, \mADtwo \rangle; \textrm{``\phantom{}''}) \\
        \mKtwoe = \mHkdfExpand(\mPRKtwo, \mTHtwo)
      \end{array}$};
    \msg{21.7em}{res}{ini}{\mMsgtwo: \mCi, \mGy, \mCr, $\overbrace{\mKtwoe\ \mXor\ \langle \mIdcredr, \mMactwo, \mADtwo \rangle}^{\mCipher}$};
    \action{22.5em}{ini}{$
      \begin{array}{c}
        %\mTHtwo = \mHash(\mMsgone, \langle \mCi, \mGy, \mCr \rangle) \
        \ P_e = \mDH(\mX,\ \mGy{})\\
        \mPRKtwo = \mHkdfExtract(\textrm{``\phantom{}''}, P_e) \\
        %\mKtwoe = \mHkdfExpand(\mPRKtwo,\mTHtwo)\\
        %\mGrx = \mCredr^{x} \\
        \ \ P_R = \mDH(\mX,\ \mPub{R})\\
        \mPRKfour = \mPRKthree = \mHkdfExtract(\mPRKtwo, P_R) \\
        %\mKtwom = \mHkdfExpand(\mPRKthree, \mTHtwo) \\
        \mKthreeae = \mHkdfExpand(\mPRKthree, \mTHtwo) \\
        \mTHthree = \mHash(\mTHtwo, \mCipher, \mCr)\\
        \mKthreem = \mHkdfExpand(\mPRKfour, \mTHthree) \\
        \mMacthree = \mAead(\mKthreem; \langle \mIdcredi, \mTHthree, \mCredi, \mADthree \rangle; \textrm{``\phantom{}''}) \\
        \mSigthree = \mSign{I}(\langle \langle \mIdcredi, \mTHthree, \mCredi, \mADthree \rangle, \mMacthree \rangle)
      \end{array}$};
    \msg{35.5em}{ini}{res}{$\mMsgthree: \mCr, \mAead(\mKthreeae; \mTHthree; \langle \mIdcredi, \mSigthree, \mADthree \rangle$)};
    \action{36em}{res}{$
    \begin{array}{c}
        \mTHthree = \mHash(\mTHtwo, \mCipher, \mCr)\\
        \mKthreem = \mHkdfExpand(\mPRKthree, \mTHthree) \\
        \mKthreeae = \mHkdfExpand(\mPRKthree, \mTHthree)
    \end{array}$};
    \draw [line width=1mm] (-2,-14.8) -- (2,-14.8);
    \draw [line width=1mm] (7,-14.8) -- (11,-14.8);
    \end{tikzpicture}
}
    \caption{The \mSigStat{} method.
    Tuples are denoted $\langle\cdot\rangle$, and the hash function \mHash{} is
    as determined by \mSuites{}.}
\label{fig:edhocsigstat}
\end{figure*}
%------------------------------------------------------------------------- sub

