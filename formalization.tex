% !TEX root = paper.tex
The \mEdhoc{} \mSpec{} \cite{our-analysis-selander-lake-edhoc-00} claims
that \mEdhoc{} satisfies many security properties, but these are imprecisely
expressed and motivated.
%
In particular, there is no coherent adversary model.
%
It is therefore not clear in which context properties should be verified.
%
We resolve this by clearly specifying an adversary model, in which we can 
verify
properties.
%

\subsection{Formal Model}\label{sec:threat-model}
As in~\cite{Norr21}, we verify \mEdhoc{} in an extended
Dolev-Yao model~\cite{DY83}.
%
Dolev-Yao is a well-established model for symbolic verification of security
protocols.
%
Messages are modelled as terms in an algebra, and the various cryptographic
operations are assumed to be perfect, e.g., encrypted messages can only be
decrypted with the correct key, and there are no hash collisions.
%

The adversary is assumed to be in control of the communication
channel and can see all messages being communicated as part of the protocol.
%
In addition, they can interact with an unbounded number of protocol sessions,
drop, inject and modify messages at will.
%

On top of the standard Dolev-Yao model,~\cite{Norr21} allows the adversary to
access the long-term and ephemeral keys.
%
Long-term key reveal, denoted $\mRevLTK^{t}(A)$, stands for the adversary
gaining access to a party $A$'s long-term private key \mPriv{A} at a time point $t$.
%
Ephemeral key reveal, denoted $\mRevEph^{t}(A, k)$, stands for the adversary
obtaining, at time $t$, the ephemeral private key \mPrivE{A} used by party $A$
in a session where they establish key material $k$.
%
Formalizing these two capabilities allows more fine-grained control
over the access that an adversary has to these fundamentally different kinds of
keys.
%
Furthermore, we extend the results from~\cite{Norr21} by strengthening the
adversary capabilities, as detailed next.
%

%---------------------------------------------------------------------- subsub
\subsubsection{System and Adversary Model Extensions Supporting TEE.}
\label{sec:TEE:advModel}
%
We extend the adversary model of~\cite{Norr21} by introducing an interface to
allow the adversary to use the long term key material of any party, though
without directly accessing such key material.
%
This allows us to model the scenario where the adversary has gained access to
a device, but the long term key material is protected by a trusted execution
environment (TEE), and thus only accessible through the TEE interface.
%
According to the terminology of~\cite{DBLP:conf/csfw/Cohn-GordonCG16}, a
protocol is considered to have \emph{weak post-compromise security}
(Weak-PCS) if it achieves its security goals for a specific session
even when the attacker has limited access to the long-term key material
of the parties involved before the session starts -- through an interface 
that maintains the key material secure but allows to run cryptographic 
operations with it -- and full access to the key material of the parties
involved after the end of the session, as well as access to the key material
of all other parties.
%
Seen in the framework of~\cite{DBLP:conf/icics/XuZRWTZ20}, we add adversary
capabilities corresponding to a server adversary, i.e., an adversary that
compromises a party, learns its ephemeral keys and may temporarily access its
TEE, but does not learn the party's long-term key.
%
Our extension to the model of~\cite{Norr21} thus allows us to verify all the
previous security properties under a Weak-PCS model.
%
Note that this attacker model is strictly more powerful than that of~\cite{Norr21},
as it maintains all the previous attacker capabilities.
%

We split the \mEdhoc{} functionality as follows.
%
The TEE contains the long-term key and allows the non-TEE parts of the
application to perform the operations on it via its interface.
%
More precisely, parties using the \mSig{} authentication method use a TEE 
with an interface which accepts a message and returns the signature of that 
message using the party's private long-term key.
%
A party $U$ using the \mStat{} authentication method use a TEE with an 
interface accepting a point $P$ on the curve and returning $\mDH(P,\ \mGu{})$.
%

The non-TEE parts implement \mEdhoc{} using this interface.
%
This interface requires the least functionality from the TEE, reducing the 
TEE's complexity and possibly cost of its implementation.
%
This functional split is suitable even when a constrained device has
implemented only the storage of the long-term key in 
a special purpose circuit with minimal processing functionality.
%
Since \mEdhoc{} focuses on constrained IoT devices, it seems appropriate to
cater for this setting.
%

%---------------------------------------------------------------------- subsub
\subsubsection{Extended Formalism and Security Properties.}
\label{sec:TEE:fmAndProps}
Formally, we model the TEE interface by adding two new rewrite rules:
%
\begin{small}
\begin{verbatim}
rule forge_SIG:
   [!LTK_SIG($A, ~ltk), In(xx)] --[TEE($A)]-> [Out(sign(xx, ~ltk))]

rule exp_STAT:
   [!LTK_STAT($A, ~ltk), In('g'^~xx)] --[TEE($A)]-> [Out(('g'^~xx)^~ltk)]
\end{verbatim}
\end{small}
%
These rules allow the adversary to obtain terms representing signatures on a
value of their choice to forge signatures (\verb|forge_SIG|), or to obtain terms
representing a curve point of their choice raised to the power of the
long-term key (\verb|exp_STAT|).
%

Because it is a trivial attack when the adversary access these rules with values
from the test session, we must disqualify those rule applications.
%
We do so by creating an action fact \verb|TEE($A)|, where \verb|$A| is the
identity corresponding to the private key used, and then augmenting the
properties with a condition that no such action fact exists from the start of
the protocol execution and its end.
%
Care must be taken when specifying the start and the end: specifically,
the start and end of the protocol run must be viewed w.r.t. the current
role.
%
For example, the injective agreement property for the initiator in
the \mSigSig{} method requires that the adversary does not have access to the
TEE of the responder from the time of transmission of the first message until
the second message is received by the initiator.
%
Because reception of one message and transmission of the next one at a party
is an atomic operation, these timepoints represent the start and end of the
protocol run from the perspective of the initiator.
%

With this additional attacker capability Tamarin manages to prove
almost all lemmas and method combinations, except for the implicit
authentication lemmas when the responder uses the \mStat{} authentication
method: in these two cases Tamarin did not complete the within reasonable
time\footnote{See Table~\ref{tab:props} for all verification results and
computation times.}.
%
We therefore simplified the modeling of XOR encryption of the second
message to prove implicit authentication when the responder uses the 
\mStat{} method.
%
More precisely, the second message encrypts two values: the responder's identity
\verb|R| and the authentication information \verb|authR|.
%
We model XOR encryption by xor-ing each term with their own key-stream term.
%
However for the problematic cases we xor the entire tuple
\verb|<R, authR>| with a single key-stream term.
%
This simplification might miss an attack on implicit authentication that is due
to the combination of a malleable XOR encryption and access to the TEE interface.
%
However, given that no attacks were identified using our original modeling due
to the use of XOR, nor in the current modeling for all other authentication 
methods, we believe that this is not a severe restriction.
%

Next we describe \mTamarin{}, and our modeling of \mEdhoc{}.
%

%-------------------------------------------------------------------------- sub
\subsection{\mTamarin{}}
\label{sec:tamarin}
We extend the formal \mEdhoc{} model of~\cite{Norr21}, using the symbolic
model tool \mTamarin{}~\cite{DBLP:conf/cav/MeierSCB13}, which is an 
interactive
tool for the formal verification of security protocols.
%
Protocols are modelled in \mTamarin{} as multiset rewrite rules which encode 
a
transition relation.
%
The elements of these multisets, called facts, contribute to the global
system state.
%
For syntactic sugar, \mTamarin{} also allows the use of let-bindings and tuples.
%
For ease of presentation, we will present the model and properties in a
slightly different syntax in this paper, but this syntax can be directly
mapped to that of \mTamarin.
%

Rewrite rules can be annotated with events, called actions in \mTamarin{}.
%
Communicated messages in the protocol are modelled as terms in an algebra,
which specifies sets of names, variables, and allowable function symbols.
%
Facts and actions are modelled as $n$-ary predicates in the term algebra,
and actions can be parametrized using terms.
%

An annotated multiset rewrite-rule is represented as
$l \ifarrow[e] r$, where $l$ and $r$ are multisets, and $e$ is
a multiset of actions.
%
A sequence of actions yields a protocol execution trace.
%
Event types are predicates over global states generated during protocol 
execution.
%
Consider an event type $E$ and a timestamp $t$ as part of a trace.
%
By $E^{t}(p_i)_{i\in\mathbb{N}}$, we represent an event of type $E$ occurring
at time $t$ in a trace, parametrized by the sequence of values
$(p_i)_{i\in\mathbb{N}}$
(corresponding to the action fact $E(p_i)_{i\in\mathbb{N}}@t$ in \mTamarin).
%
Thus, the time points form a quasi order, and we denote the fact that
$t_{1}$ comes before $t_{2}$ in a protocol trace by
$t_{1} \lessdot t_{2}$, and that $t_{1}$ and $t_{2}$ stand for the same
time point in a trace by $t_{1} \doteq t_{2}$.
%
\mTamarin{} allows events to occur at the same time point, with one
restriction: multiple events of the same type cannot occur simultaneously,
so if $t_{1} \doteq t_{2}$, then $E^{t_{1}} = E^{t_{2}}$.
%

Properties are defined as formulas in a fragment of temporal first order logic,
and these formulas can be verified over execution traces.
%

Protocol verification in \mTamarin{} happens under an equational theory $E$.
%
For example, to represent the fact that $E$ satisfies the reversal of
symmetric encryption by using a decryption operation with the key,
one can write $\textit{dec}(\textit{enc}(x, y), y) =_{E} x$.
%
The equational theory $E$ is fixed upfront to handle the functions supported 
by
the term algebra, so we will omit the subscript for the rest of this paper.
%

Users can extend the default term algebra and equational theory in
\mTamarin{} with new function symbols and unification rules for these new 
symbols.
%
For example, \mEdhoc{} requires authenticated encryption, which we model 
using
the symbol \mT{aeadEncrypt}.
%
We augment \mTamarin{} with the following rule for this operation,
which represents the fact that if the adversary knows a key \mT{k},
a message \mT{m}, and authenticated data \mT{ad},
and has access to an encryption algorithm \mT{ai},
then they can obtain the message corresponding to the authenticated 
encryption
of \mT{m} with \mT{k}~\cite{Norr21}.
%
{\footnotesize
\begin{verbatim}
[!KU(k), !KU(m), !KU(ad), !KU(ai)] --[]-> [!KU(aeadEncrypt(k, m, ad, ai))]
\end{verbatim}}
%
Other than modeling authenticated encryption, we use \mTamarin{}'s builtin
equational theories for signing, Diffie-Hellman, hashing and \mXor{}.
%

Tamarin has also built-in rules for modeling a Dolev-Yao adversary and the
evolution of their knowledge as the protocol executes.
%
We extend the Dolev-Yao adversary model by adding other rules that increase
the capabilities of the attacker.
%
To denote that the adversary has access to a message $p$ at time $t$, we use 
$\mK^{t}(p)$.
%
\anote{Removed:
This stands for the fact that as the adversary interacts with the parties
executing the protocol, at time point $t$, the adversary can derive the
term $p$ using the Dolev-Yao deduction rules and any additional adversary
capabilities specified in \mTamarin.}
%
As an example, the following implication
\[
    \forall t, k, k'\mLogicDot \mK^{t}(\langle k, k'\rangle)\ \rightarrow\
\mK^{t}(k) \land \mK^{t}(k'),
\]
models that if the adversary gets to know the pair of
keys $\langle k, k' \rangle$ at a time point $t$, then the adversary also
knows each of those keys $k$ and $k'$ at time point $t$ as well.
%
For more details about how \mTamarin{} manages adversary knowledge,
see~\cite{DBLP:conf/cav/MeierSCB13}.
%

%-------------------------------------------------------------------------- sub
\subsection{Model and Desired Properties}
\label{sec:desired-properties}
In this section we describe our modeling of \mEdhoc{} and its desired
security properties.
%

The party $I$ executing the initiator role considers a run of the protocol
begun as soon as it sends the first message \mMsgone{} with event type
\mIStart, and considers the run ended once it has sent the third message
\mMsgthree{} with event type \mIComplete.
%
Similarly, the responder $R$ considers a run started upon receiving 
\mMsgone{}
with event type \mRStart, and finished upon receiving \mMsgthree{} with 
type
\mRComplete.
%

In this work we consider the following properties: secrecy of the session key,
injective agreement and implicit agreement of the session key material for both
initiator and responder, and secrecy and integrity of the application data sent
on message 3 ($\mAD_3$)
%
Agreement is considered on a set of parameters $S$ which also contains the
session key material \mSessKey{}.
%
We will first describe in detail both properties, and then describe the
contents of the set $S$.
%
We formalize these properties as shown in Figure~\ref{fig:props}, which is
a modified version of a figure in~\cite{Norr21}.
\begin{figure*}[ht]
\begin{align*}
    \mPredPcs \triangleq\ & \forall \mTID, I, R, \mSessKey, t_2, t_3\mLogicDot
    \mK^{t_3}(\mSessKey)\  \land\ 
    (\mIComplete^{t_2}(\mTID, I, R, \mSessKey)\, \lor\, \mRComplete^{t_2}(\mTID, I, R, 
\mSessKey))
    \rightarrow\\
    &(\exists t_1\mLogicDot \mRevLTK^{t_1}(I) \land t_1 \lessdot t_2)
    \ \lor\ (\exists t_1\mLogicDot \mRevLTK^{t_1}(R) \land t_1 \lessdot t_2)\\
    &\ \lor\ (\exists t_1\mLogicDot \mRevEph^{t_1}(R, \mSessKey))
    \ \lor\ (\exists t_1\mLogicDot \mRevEph^{t_1}(I, \mSessKey))\\
    &\ \lor\ (\exists t_0, t_1\mLogicDot \mIStart^{t_0}(\mTID, I, R) \land \mTEE^{t_1}(R) \land t_0 \lessdot t_1)\\
    &\ \lor\ (\exists t_0, t_1\mLogicDot \mRStart^{t_0}(\mTID, R, \mSessKey, S) \land \mTEE^{t_1}(I) \land t_0 \lessdot t_1)\\[2em]
%
    \mPredInjI \triangleq\ &
    \forall \mTID_I, I, R, \mSessKey, S, t_2\mLogicDot \mIComplete^{t_2}(I, R, 
\mSessKey, S)
    \rightarrow\\
    &(\exists \mTID_R, t_1\mLogicDot \mRStart^{t_1}(\mTID_R, R, \mSessKey, S) \land t_1 \lessdot t_2
    \land (\forall \mTID_I', I', R', t_1' \mLogicDot \mIComplete^{t_1'}(\mTID_I', I' , R', \mSessKey, S)
        \rightarrow t_1' \doteq t_1))\\
    &\ \lor\ (\exists t_1\mLogicDot \mRevLTK^{t_1}(R) \land t_1 \lessdot t_2)\\
    &\ \lor\ (\exists t_0, t_1\mLogicDot \mIStart^{t_0}(\mTID_I, I, R) \land \mTEE^{t_1}(R) \land t_0 \lessdot t_1)\\[2em]
%
    \mPredInjR \triangleq\ &
    \forall \mTID_R, I, R, \mSessKey, S, t_2\mLogicDot \mRComplete^{t_2}(\mTID_R, I, R, 
\mSessKey, S)
    \rightarrow\\
    &(\exists \mTID_I, t_1\mLogicDot \mIStart^{t_1}(\mTID_I, I, R, \mSessKey, S) \land t_1 \lessdot t_2
    \land (\forall \mTID_R', I' R' t_1' \mLogicDot \mRComplete^{t_1'}(\mTID_R', I' , R', \mSessKey, S)
        \rightarrow t_1' \doteq t_1))\\
    &\ \lor\ (\exists t_1\mLogicDot \mRevLTK^{t_1}(I) \land t_1 \lessdot t_2)\\
    &\ \lor\ (\exists t_0, t_1\mLogicDot \mRStart^{t_0}(\mTID_R, R, \mSessKey, S) \land \mTEE^{t_1}(I) \land t_0 \lessdot t_1)\\[2em]
%
    \mPredImpI \triangleq\ &
    \forall \mTID_I, I, R, \mSessKey, S, t_1\mLogicDot \mIComplete^{t_1}(\mTID_I, I, R, \mSessKey, 
S)
    \rightarrow\\
      &(\forall \mTID_R, I', R', S', t_2\mLogicDot \mRComplete^{t_2}(\mTID_R, I', R', \mSessKey, S') \rightarrow
             (I=I' \land R=R' \land S=S')\\
      &\ \ \land (\forall \mTID_I', I', R', S', t_1'\mLogicDot
        \mIComplete^{t_1'}(\mTID_I', I', R', \mSessKey, S') \rightarrow t_1' \doteq t_1
        ))\\
    &\lor(\exists t_0\mLogicDot \mRevLTK^{t_0}(R) \land t_0 \lessdot t_1)
    \lor(\exists t_0\mLogicDot \mRevEph^{t_0}(R, \mSessKey))
    \lor(\exists t_0\mLogicDot \mRevEph^{t_0}(I, \mSessKey))\\
    &\lor\ (\exists t_0, t_1\mLogicDot \mIStart^{t_0}(\mTID_I, I, R) \land \mTEE^{t_1}(R) \land t_0 \lessdot t_1)
\end{align*}
%
\caption{Formalization of security properties and adversary model.}
\label{fig:props}
\end{figure*}


%----------------------------------------------------------------------- PCS
\subsubsection{Secrecy of the session key material.}
\label{sec:secrecy}
We show that the adversary cannot gain access to the session key material \mSessKey{},
even under a weak post-compromise security model that allows the adversary limited access
to the long-term keys of the parties involved through a secure interface before the session starts, 
unrestricted access to the long-term keys of the involved parties after the session ends
and of all other parties, as well as access to all other session keys.
%
This property is formalized as \mPredPcs{}
in Figure~\ref{fig:props}.
%

In the \mIComplete{} event, \mTID{} represents a ``thread identifier'' for the
current session and is used to correlate multiple events for the same run of the 
protocol for each party, $I$ represents the identity of the initiator,
and $R$ represents the identity of the party who the initiator believes is
playing the responder role, while \mSessKey{} stands for the established
session key material.
%
\mRComplete{} has analogous parameters; in particular, the responder $R$
believes $I$ is the party playing the initiator role.
%
Intuitively, the PCS property states that an adversary obtains \mSessKey{}
only if one of the following conditions hold: either a party's long-term key
is compromised before their run ends, or \mSessKey{} is directly revealed to
the attacker, or the TEE interface for the responder is used during the current
session for the initiator (i.e. after the initiator starts), or, similarly, 
the TEE interface for the initiator is used during the
current session for the responder.
%
We use the thread identifier \mTID for each party to relate the starting event
to the corresponding commit event.
%
The property in Figure~\ref{fig:props} is unlike the actual \mTamarin{} lemma
in one minor way: \mTamarin's logic does not allow disjunctions to appear on
the left-hand side of an implication inside a universally-quantified formula.
%
Therefore, in the \mTamarin{} code, instead of using the disjunction
$\mIComplete^{t_2}(\mTID, I, R, \mSessKey)\, \lor\, 
\mRComplete^{t_2}(\mTID, I, R, \mSessKey)$
to model the fact that either party may have completed their execution, we use
a single action parametrized by the terms $\mTID$, $I$, $R$, and \mSessKey.

%-------------------------------------------------------------------- InjAgree
\subsubsection{Authentication Properties.}
\label{sec:authenticationDef}
Following~\cite{Norr21}, we prove two different kinds of authentication
properties, namely \emph{injective agreement} in the style
of~\cite{DBLP:conf/csfw/Lowe97a}, and implicit agreement.
%
Injective agreement can be guaranteed to either party running the protocol.
%
For the initiator $I$, it guarantees to $I$ that whenever $I$ believes that
they have completed a run with $R$ as responder, then the party $R$ has 
indeed
executed the protocol in the role of a responder, and that this run of $I$
uniquely corresponds to one of $R$ where the set of parameters is $S$ and
includes, in particular, the session key material \mSessKey{}.
%
It can be defined for $R$ in a similar manner.

%
We formalize injective agreement for the initiator role as \mPredInjI{} and
for the responder role as \mPredInjR{} in Figure~\ref{fig:props}.
%
For the initiator $I$, this property represents the fact that either
injective agreement (as described above) holds, or the long-term key of
the party $R$ assumed to be playing the responder role has been
compromised before $I$'s role finished.
%
The property should hold even if the adversary has access to the TEE of party $R$
before and after the protocol execution.
%
%If the adversary has managed to compromise $R$'s long-term key, they can
%generate any message of their liking and sign it using said key,
%or compute a $\mathit{MAC}_{R}$ using the long-term keys \mPubE{I},
%\mPriv{R}, and an ephemeral key pair $\langle\mPrivE{R},\ 
%\mPubE{R}\rangle$
%of 
%their choice.
%%
An analogous definition holds for the responder $R$.
%

%------------------------------------------------------------- Implicit auth
As part of our analysis, we found that all the \mEdhoc{} methods satisfy weak
pot-compromise security.
%
However, this is not the case for the injective agreement property as stated 
above.
%
Thus, we show a different property, a form of implicit agreement on the same
set of parameters, which is guaranteed for all methods.
%
This modification is inspired by the definitions of implicit authentication in
the computational model~\cite{DBLP:conf/csfw/GuilhemFW20}.
%
While that paper focuses on authenticating just a key and related identities,
our definition encompasses a general set of parameters, as in the notion of
injective agreement proposed by Lowe~\cite{DBLP:conf/csfw/Lowe97a}.

The ``implicit'' in the name of the property stands for the fact that a party
$A$ assumes that any party $B$ who has access to the session key material
\mSessKey{} must, in fact, be the intended party, and that if $B$ is honest,
$B$ will agree on a set $S$ of parameters which includes \mSessKey.
%
Implicit agreement for both roles guarantees to $A$ that $A$ is or has been
involved in exactly one protocol execution with $B$, and that $B$ agrees or
will agree with $A$ on $S$.
%
This property diverges from injective agreement in that upon sending
the last message, $A$ concludes that if this message reaches $B$, then $A$
and $B$ agree on each other's identities and roles, and the set $S$.
%

Note that for implicit agreement to hold for $I$, the ephemeral keys must not
be revealed since the property relies on the fact that the intended responder
is the only one who knows the session key material.
%
If the adversary has access to the ephemeral keys, they can use them along 
with
the public keys of $I$ and $R$ to compute the session key material.
%
However, either party's long-term key can be revealed after the other
party has finished their execution, since this still leaves the adversary
unable to compute \mGxy{}.
%

Since \mTamarin{} runs out of memory to verify this property as is,
we split it into two lemmas -- \mPredImpI{} for $I$ for one \mPredImpR{} for 
$R$.
%
Figure~\ref{fig:props} contains the definition for \mPredImpI{}.
%
\mPredImpR{} is formalized similarly, so we omit it.
%

%------------------------------------------------------- Agreed parameters
\subsubsection{Parameters in set $S$.}
\label{sec:agreedParams}
We now describe the set $S$ of parameters upon which the two parties obtain
guarantees via the above properties.
%
The initiator $I$ gets injective and implicit agreement guarantees on the
following partial set $S_P$ of parameters~\cite{Norr21}:
\begin{itemize}
    \item the roles played by itself and its peer,
    \item responder identity,
    \item session key material (which varies depending on the \mEdhoc{} 
method),
    \item context identifiers \mCi{} and \mCr{}, and
    \item cipher suites \mSuites{}.
\end{itemize}
%

Due to the initiator being guaranteed identity protection under \mEdhoc{}, $I$
cannot get explicit agreement with $R$ on the initiator's identity.
%
Similarly, when using the \mStat{} authentication method, $I$ does not get 
any
such guarantees about $P_{I}$.
%
However, $I$ does get implicit agreement with $R$ about $I$'s identity and the
full set $S_{F}$ of agreed parameters.
%
In contrast, since $R$'s run finishes after that of $I$, $R$ can get explicit
injective agreement assurances on the full set $S_{F}$ of agreed parameters.
%
The full set of agreed parameters $S_F$ is $S_P \cup \{I, P_I\}$ when $P_I$
is part of the session key material, and $S_P \cup \{I\}$ otherwise.
%

In addition to these properties, a couple of properties can be inferred
without being explicitly modelled and verified.
%
One such property is Key-Compromise Impersonation
(KCI)~\cite{DBLP:conf/ima/Blake-WilsonJM97}.
%
A KCI attack occurs when an adversary who has access to $A$'s long-term 
private
key to make $A$ believe that they completed an execution with a peer $B$,
while $B$ did not participate in said execution at all.
%
This is in particular relevant when \mStat{} authentication methods are used.
%
Our above notions of agreement ensure that both parties agree on each
other's identity, role, and session key material.
%
Therefore, all \mEdhoc{} methods that satisfy these agreement properties also
avoid KCI attacks.
%

Another kind of attack is Unknown Key Share attacks
(UKS)~\cite{DBLP:conf/ima/Blake-WilsonJM97}.
%
As part of a UKS attack, a party $A$ can be made to believe that it finished
an execution with a party $B$, but where the session key material is actually
shared between $A$ and $C$ instead.
%
Again, due to the agreement on identities and session key material, any 
method
that satisfies the above agreement properties also resists UKS attacks.
%
Overall, the agreement properties capture entity authentication,
and satisfy any properties based on that notion.
%
However, see Section~\ref{sec:unintendedPeerAuth} for a discussion on the
interaction between \mEdhoc{} and the application leading to similar issues.
%

%---------------------------------------------------------------------------sub
\subsection{Encoding \mEdhoc{} in \mTamarin}
\label{sec:modeling}
%
In this section, we describe how we model the \mSigSig, \mSigStat, \mStatSig,
and \mStatStat{} methods of \mEdhoc{} in \mTamarin.
%
As in~\cite{Norr21}, we construct the \mTamarin{} model by utilizing the fact
that all methods of \mEdhoc{} share a common underlying structure
(as shown in Figure~\ref{fig:edhocFramework}).
%
We do so by using a single specification file in the M4 macro language,
which generates all the methods.
%
As in~\cite{Norr21}, we only present the \mStatSig{} method which illustrates
the use of two different authentication methods.
%
The full \mTamarin{} code for all models can be found 
at~\cite{edhocTamarinRepo}.
%

As mentioned in Section~\ref{sec:tamarin}, we extend the default
equational theory of \mTamarin{} to handle various operations used in 
\mEdhoc.
%
Of the built-in theories, we use the ones for exclusive-or (\mXor),
Diffie-Hellman operations, signatures (\mT{sign} and \mT{verify} operations),
and 
hashing~\cite{DBLP:conf/csfw/DreierHRS18,DBLP:conf/csfw/SchmidtMCB12}.
%
We 
%augment the hashing function symbol to add an extra input, which we use 
to
model different hash functions.
%

In addition to these default operations, \mEdhoc{} is built over
\mHkdfExpand, \mHkdfExtract, and the \mAead{} functions.
%
We represent \mHkdfExpand{} by \mT{expa} and \mHkdfExtract{} by 
\mT{extr}.
%
\mAead{} operations need us to add extra equations to the underlying theory.
%
A term encrypted using \mAead{} is represented by \mT{aeadEncrypt(m, k, ad, 
ai)},
where \mT{m} is the underlying message, \mT{k} is the encrypting key,
\mT{ad} is the additional data, and \mT{ai} is the identifier for the
encryption algorithm.
%
Decryption of such a term is defined via two equations that we add to
\mTamarin's theory.
%
The following equation requires the decrypting party to know the additional
data \mT{ad} to decrypt this encrypted term with verification of its integrity.
\begin{small}
\begin{verbatim}
aeadDecrypt(aeadEncrypt(m, k, ad, ai), k, ad, ai) = m.
\end{verbatim}
\end{small}
%
Only the above equation is used by honest parties, but the adversary should
also be able to decrypt without having to go through the additional step of
identity verification.
%
To this end, we also add the following equation, where the adversary does not
need access to the additional data \mT{ad} in order to decrypt.
%
\begin{small}\begin{verbatim}
decrypt(aeadEncrypt(m, k, ad, ai), k, ai) = m.
\end{verbatim}\end{small}
%

Having described how we adapt the equational theory to model \mEdhoc,
we now move on to the modelling of the adversary model and the 
environment in
which the protocol is executed.
%
We extend the built-in Dolev-Yao adversary rules which are part of 
\mTamarin.
%
We use the following rules to capture the link between a party's identity and
their long-term key pairs, in the \mSig{}- and in the \mStat{}-based methods
respectively.
\begin{center}
\begin{minipage}{0.48\textwidth}
\begin{scriptsize}
\begin{verbatim}
1 rule registerLTK_SIG:
2    [Fr(~ltk)] --[UniqLTK($A, ~ltk)]->
3        [!LTK_SIG($A, ~ltk),
4         !PK_SIG($A, pk(~ltk)),
5         Out(<$A, pk(~ltk)>)]
\end{verbatim}
\end{scriptsize}
\end{minipage}
\hfill\vline\hfill
\begin{minipage}{0.48\textwidth}
\begin{scriptsize}
\begin{verbatim}
1 rule registerLTK_STAT:
2    [Fr(~ltk)] --[UniqLTK($A, ~ltk)]->
3        [!LTK_STAT($A, ~ltk),
4         !PK_STAT($A, 'g'^~ltk),
5         Out(<$A, 'g'^~ltk>)]
\end{verbatim}
\end{scriptsize}
\end{minipage}
\end{center}

Using the fact \verb|Fr(~ltk)|, \mTamarin{} creates a new term \mT{ltk} and 
uses
it to represent a secret long-term key.
%
Via the \verb|Out(<$A, pk(~ltk)>)| fact, \mTamarin{} puts out onto the
communication channel the identity of the party to whom this long-term
key belongs, along with their public key.
%
Since the adversary has access to the communication channel, they can pick 
up
all of this information.
%
The event \mT{UniqLTK} parametrized by a party's identity and their 
long-term
key 
models the unique correspondence between those two values.
%
As a result, this rules out a party owning multiple long-term keys
-- in particular, it keeps the adversary from registering long-term keys in
some honest party's name.
%
This aligns well with an external mechanism such as a certificate authority
ensuring that long-term keys are uniquely assigned to the corresponding
identities, which is ensured by the \mSpec.

To model the reveal of long-term keys and ephemeral keys to an adversary,
we use standard reveal rules and events of type \mRevLTK{} and \mRevEph,
respectively.
%
It is also important to keep track of the time points at which these events 
occur.
%
Long-term keys can be revealed on registration, even before the protocol 
begins.
%
Ephemeral keys in our model can only be revealed when a party
completes their role, i.e., simultaneously with events of type \mIComplete{}
and \mRComplete.
%
Having set out the capabilities of the adversary, we now model the execution
of the honest agents' roles.
%

For each protocol method, we use two rules apiece for the initiator and
responder -- \mT{I1}, \mT{R2}, \mT{I3}, \mT{R4}.
%
Each of these stand for one step of the protocol as executed by either party.
%
To disambiguate, we will attach the method to the rule name.
%
These four rules directly map to the event types
\mIStart, \mRStart, \mIComplete, and \mRComplete, respectively.
%
We show the \mT{R2\_STAT\_SIG} rule below to illustrate the various aspects
of the \mTamarin{} modeling we are describing here.
%

In order to keep track of the initiator's state, we use facts prefixed with
\mT{StI}, which carry information between the \mT{I1} and \mT{I3} rules.
%
Similarly, for the responder's state, we have \mT{StR} to carry state data
between \mT{R2} and \mT{R4}.
%
In order to link two rules to a state fact, we use \mT{tid}, which
is unique to the current session.
%
The use of these state facts can be seen in line 28 in the \mT{R2\_STAT\_SIG} 
rule.
%

Note that we do not model any error message that $R$ might send in response
to message \mMsgone rejecting $I$'s choice of ciphersuite and/or method.
%

As in~\cite{Norr21}, we model the \mXor{} encryption of 
\mT{CIPHERTEXT\_2} with
the key \mT{K\_2e} by allowing each part of the encrypted term to be
separately attacked.
%
This means that we first expand \mT{K\_2e} to the same number of key terms 
as
subterms in the plaintext tuple.
%
This is done by applying \mHkdfExpand{} to unique inputs per subterm.
%
After this, we \mXor{} each subterm with its own key term.
%
This is more faithful to the \mSpec{} than \mXor-ing \mT{K\_2e} on its own
with the plaintext tuple.
%
This can be seen in lines 19--22 in the code for \mT{R2\_STAT\_SIG}.
%

As we extended the model with TEEs and augmented the adversary's 
capabilities
with access to them, \mTamarin{} failed to complete in a reasonable time for
some combination of authentication methods and security properties (see 
Section~\ref{sec:conclusion} for a detailed discussion).
%
To circumvent the problem, we simplified the \mXor{} encryption to 
\mXor-ing 
a
single term on the entire tuple for these cases. \\
%

{\parindent 0pt
\begin{minipage}{\textwidth}
\begin{scriptsize}
\begin{verbatim}
1 rule R2_STAT_SIG:
2 let
3    agreed = <CS0, CI, ~CR>
4    gx = 'g'^xx
5    data_2 = <'g'^~yy, CI, ~CR>
6    m1 = <'STAT', 'SIG', CS0, CI, gx>
7    TH_2 = h(<$H0, m1, data_2>)
8    prk_2e = extr('e', gx^~yy)
9    prk_3e2m = prk_2e
10   K_2m = expa(<$cAEAD0, TH_2, 'K_2m'>,
11               prk_3e2m)
12   protected2 = $V // ID_CRED_V
13   CRED_V = pkV
14   extAad2 = <TH_2, CRED_V>
15   assocData2 = <protected2, extAad2>
16   MAC_2 = aead('e', K_2m, assocData2, $cAEAD0)
17   authV = sign(<assocData2, MAC_2>, ~ltk)
18   plainText2 = <$V, authV>
19   K_2e = expa(<$cAEAD0, TH_2, 'K_2e'>, prk_2e)
20   K_2e_1 = expa(<$cAEAD0, TH_2, 'K_2e', '1'>, prk_2e)
21   K_2e_2 = expa(<$cAEAD0, TH_2, 'K_2e', '2'>, prk_2e)
22   CIPHERTEXT_2 = <$V XOR K_2e_1, authV XOR K_2e_2>
23   m2 = <data_2, CIPHERTEXT_2>
24   exp_sk = <gx^~yy>
25 in
26   [!LTK_SIG($V, ~ltk), !PK_SIG($V, pkV), In(m1), Fr(~CR), Fr(~yy), Fr(~tid)]
27   --[ExpRunningR(~tid, $V, exp_sk, agreed), R2(~tid, $V, m1, m2)]->
28   [StR2_STAT_SIG($V, ~ltk, ~yy, prk_3e2m, TH_2, CIPHERTEXT_2, gx^~yy, ~tid, m1, m2, agreed),
29    Out(m2)]
\end{verbatim}
\end{scriptsize}
\end{minipage}}
\vspace{5mm}

As mentioned earlier, we use actions to represent parametrized events.
%
For example, in line 27 above, the action
\verb|ExpRunningR(~tid, $V, exp_sk, agreed)| represents an event of type
\mRStart{} parametrized by the session id, the responder's identity, and the
session key material \mT{exp\_sk}.
%
The \mT{exp} in the name of the variable for session key material represents
the fact that the agreement property satisfied by this key is explicit, i.e.,
it includes $P_{I}$, as in Section~\ref{sec:agreedParams}.
%
We use \mT{imp\_sk} for the corresponding session key material which does 
not
include $P_{I}$.
%
For the \mSigSig{} and \mSigStat{} methods, therefore, the two values are the 
same.
%

%-------------------------------------------------------------------------- sub
%\subsection{Encoding the Desired Properties in \mTamarin}
%\label{sec:propertyFormalization}
%The properties and adversary model we defined in
%Section~\ref{sec:desired-properties} translate directly into \mTamarin's logic,
%using the straightforward mapping of events to the actions emitted from the 
%model.
%

The properties we listed in Section~\ref{sec:desired-properties}
translate directly into \mTamarin's logic.
%
We show the \mTamarin{} lemma which encodes the \mPredPcs{} property.
%
Other properties are formalized similarly. 
%

\begin{small}
\begin{verbatim}
1 lemma secrecyPCS:
2    all-traces
3    "All u v sk tid #t3 #t2.
4        (K(sk)@t3 & CompletedRun(u, v, sk, tid)@t2) ==>
5        ( (Ex #t1. LTKRev(u)@t1 & #t1 < #t2)
6        | (Ex #t1. LTKRev(v)@t1 & #t1 < #t2)
7        | (Ex #t1. EphKeyRev(sk)@t1)
8        | (Ex m1 #t0 #k. I1(tid, u, v, m1)@t0 & TEE(v)@k & t0 < k)
9        | (Ex m1 m2 #t0 #k. R2(tid, v, m1, m2)@t0 & TEE(u)@k & t0 < k)
10       )"
\end{verbatim}
\end{small}
%

In this formalization, we use the action \mT{CompletedRun(u, v, sk)}
(in line 4) to represent the disjunction of the events $\mIComplete^{t_{2}}$
and $\mRComplete^{t_{2}}$.
%
As expected, this action is emitted by both \mT{I3} and \mT{R4}.
%
Similarly, the action \mT{EphKeyRev(sk)} in line 7 stands for the reveal of
the ephemeral key for $I$ or $R$ or both.
%
Lines 8 and 9 captures that the parties TEEs must be inaccessible to the
adversary after the start of the session as seen by each party
respectively.
%
