% !TEX root = paper.tex
IoT protocols are often run on devices which operate under severe restrictions
on resources like bandwidth and energy consumption.
%
These constrained devices are often simple in their operation, but need to
communicate and function without human interference or maintenance for
extended periods of time.
%
The Internet Engineering Task Force (IETF) is standardizing new protocols to secure communications between devices that operate under such restrictions.
%
One such is the Object Security for Constrained RESTful Environments
(\mOscore{}) protocol.
%
However, \mOscore{} requires the pre-establishment of a security context.
%
To this end, a key exchange protocol named Ephemeral Diffie-Hellman Over
COSE
(\mEdhoc{}) is being discussed in the IETF.
%
Since \mEdhoc{} will establish security contexts for \mOscore{}, the same
resource constraints (especially those pertaining to message size) apply to the
former as for the latter.
%
While establishing security contexts for \mOscore{} is the primary goal for the
\mEdhoc{} protocol, there might well be other use cases which have not been
explored in depth yet.
%
It is therefore important to ensure that fundamental properties
expected of key exchange protocols as established in the literature are satisfied
by \mEdhoc{} as well.
%

%-------------------------------------------------------------------------- sub
\subsection{Evolution of \mEdhoc}
\label{sec:edhocevol}
The first \mEdhoc{} framework was introduced in March 2016.
%
It allowed two different key establishment methods -- one involved a pre-shared
Diffie-Hellman (DH) \emph{cryptographic core}, and the other was a
variation on challenge-response signatures, {\`a} la
\mOptls{}~\cite{DBLP:conf/eurosp/KrawczykW16}.

%
A \emph{cryptographic core}, often just called a core, is an academic protocol,
i.e., with no encodings or application-specific details as needed for an
industrial protocol.
%
Once these ingredients are added to a cryptographic core, we obtain a
key-establishment method.
%
Since then, the protocol has seen multiple changes.
%
In May 2018, the designers replaced the challenge-response signature core with
one based on \mSigma{}
(SIGn-and-MAc)~\cite{sigma,bruni-analysis-selander-ace-cose-ecdhe-08}, and
in
2020, three new cores, which mixed challenge-response signatures and regular
signatures were added as well~\cite{our-analysis-selander-lake-edhoc-00}.
%
While this is the version we base our formal analysis on, there have been later versions. 
%
See Section~\ref{sec:newdrafts} for a more detailed discussion.

%-------------------------------------------------------------------------- sub
\subsection{Related Work and Contributions}
\label{sub:related}
The earliest related work is~\cite{DBLP:conf/secsr/BruniJPS18}, which formally
analyzes the May 2018 version of \mEdhoc{} using the \mProverif{}
tool~\cite{DBLP:conf/csfw/Blanchet01}.
%
In this paper, the authors analyze two key establishment methods -- one built
on
a pre-shared key authentication core, and one based on \mSigma{}.
%
The authors check  the following properties: secrecy, identity protection, strong authentication, perfect forward secrecy
(PFS), and integrity of application data.
%
Later work~\cite{Norr21} analyzes the July 2020 version of \mEdhoc{} in the
\mTamarin{} prover~\cite{DBLP:conf/cav/MeierSCB13}.
%
This version of the protocol has four key establishment methods.
%
In~\cite{Norr21}, the properties checked for are injective agreement, implicit
agreement, and perfect forward secrecy for the session key material.
%
That work also includes a discussion about the various design choices made as
part of \mEdhoc{}, and the impact of \mEdhoc{} in multiple use-case scenarios.
%%

%-------------------------------------------------------------------------- sub
\subsection{Contributions}
\label{sec:contributions}
In this paper, we extend the work presented in~\cite{Norr21}.
%
We formally analyze the \mEdhoc{} specification as of July
2020~\cite{our-analysis-selander-lake-edhoc-00}, but our formal analysis applies as far as the version of the \mSpec{} from February 2021.
%
We extend the adversary model and \mTamarin{} system models to capture weak
post-compromise security (PCS) and model Trusted Execution Environments
(TEE).
%
We also alter the formal modelling used for encryption under \mXor{}.
%
We formally verify the following properties:
\begin{itemize}
\item Injective agreement
\item Implicit agreement
\item Perfect Forward Secrecy (PFS) for the session-key material
\item Weak post-compromise security for the session-key material
\item Secrecy and integrity of \mADthree
\end{itemize}
%
We follow the definition of weak PCS by~\cite{DBLP:conf/csfw/Cohn-GordonCG16},
which subsumes PFS.
%
Since \mEdhoc{} is still a protocol under development, the latest version of \mEdhoc{} is the one from July 2022~\cite{draft-ietf-lake-edhoc-15}, which contains details which are not covered by our formal model.
%
We do, however, refer to various aspects
(error handling, denial of service etc) of the July 2022 version of the \mSpec{}, namely 
in Sections~\ref{sec:errorHandling} and \ref{sec:discussion}.
%
We also discuss various issues arising due to the use of trusted
execution environments (TEEs), denial of service (DoS) attacks, error handling,
the negotiation of parameters (which the formal model abstracts
away) for establishing the protocol, and other potential attacks and concerns.
%
We have communicated these issues to the developers of the protocol.
%