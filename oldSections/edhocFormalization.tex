The \mEdhoc{} \mSpec{} \cite{our-analysis-selander-lake-edhoc-00} claims
that \mEdhoc{} satisfies many security properties, but these are imprecisely
expressed and motivated.
%
In particular, there is no coherent adversary model.
%
It is therefore not clear in which context properties should be verified.
%
We resolve this by clearly specifying an adversary model, in which we can verify
properties.
%

\subsection{Adversary Model}\label{sec:threat-model}
We verify \mEdhoc{} in the symbolic Dolev-Yao model, with idealized
cryptographic primitives, e.g, encrypted messages can only be
decrypted using the key, no hash collisions exist etc.
%
The adversary controls the
communication channel, and can interact with an unbounded number of sessions
of the protocol, dropping, injecting and modifying messages at their liking.
%

In addition to the basic Dolev-Yao model, we also consider two more adversary
capabilities, namely long-term key reveal and ephemeral key reveal.
%
Long-term key reveal models an adversary compromising a party $A$'s
long-term private key \mPriv{A} at time $t$, and we denote this event by
$\mRevLTK^t(A)$.
%
The event $\mRevEph^t(A, k)$ represents that the adversary learns
the ephemeral private key \mPrivE{A} used by party $A$ at time $t$ in a session
establishing key material $k$.
%
These two capabilities model the possibility to store and operate on
long-term keys in a secure module, whereas ephemeral keys
may be stored in a less secure part of a device.
%
This is more granular and realistic than assuming an adversary have equal
opportunity to access both types of keys.
%

We now define and formalize the security properties we are interested in, and
then describe how we encode them into \mTamarin{}-tool.
%
The adversary model becomes part of the security properties themselves.
%

\subsection{Formalization of Properties}
\label{sec:desired-properties}
We use the \mTamarin{} verification
tool~\cite{DBLP:conf/cav/MeierSCB13} to encode the model and verify properties.
%
This tool uses a fragment of temporal first order logic to reason about
events and knowledge of the parties and of the adversary.
%
For conciseness we use a slightly different syntax than
that used by \mTamarin{}, but which has a direct mapping to \mTamarin{}'s logic.
%

Event types are predicates over global states of system execution.
%
Let $E$ be an event type and let $t$ be a timestamp associated with a point in a
trace.
%
Then $E^{t}(p_i)_{i\in\mathbb{N}}$ denotes an event of type $E$ associated with
a sequence of parameters $(p_i)_{i\in\mathbb{N}}$ at time $t$ in that trace.
%
In general, more than one event may have the same timestamp and hence
timestamps form a quasi order, which we denote by $t_1 \lessdot t_2$ when $t_1$
is before $t_2$ in a trace.
%
We define $\doteq$ analogously.
%
However, two events of the same type cannot have the same timestamp, so
$t_1 \doteq t_2$ implies $E^{t_1} = E^{t_2}$.
%
Two events having the same timestamp does not imply the there is a fork in the
trace, only that the two events happen simultaneously and that comparing their
timestamps results in that they are equal.
%
This notation corresponds to \mTamarin{}'s use of action facts
$E(p_i)_{i\in\mathbb{N}}@t$.
%

The event $\mK^t(p)$ denotes that the adversary knows a parameter $p$ at
time $t$.
%
Parameters are terms in a term algebra of protocol specific operations and
generic operations, e.g., tuples $\langle\cdot\rangle$.
%
Intuitively, $\mK^t(p)$ evaluates to true when $p$ is in
the closure of the
parameters the adversary observed from interacting with parties using the
protocol, under the Dolev-Yao message deduction operations and
the advanced adversary capabilities up until time $t$.
%
For a more precise definition of knowledge management, we refer to~\cite{DBLP:conf/cav/MeierSCB13}.
%
An example of a formula is
\[
    \forall t, k, k'\mLogicDot \mK^{t}(\langle k, k'\rangle)\ \rightarrow\ 
\mK^{t}(k) \land \mK^{t}(k'),
\]
expressing that if there is a time $t$ when the adversary knows the tuple
$\langle k, k'\rangle$, then the adversary knows both
$k$ and $k'$ at the same point in time.
%

An initiator $I$ considers the
protocol run started when it sends a message \mMsgone{} (event type \mIStart)
and the run completed after sending a message \mMsgthree{} (event type
\mIComplete).
%
Similarly, a responder $R$ considers the run started upon receiving
a \mMsgone{} (event type \mRStart), and completed upon receiving a \mMsgthree{}
(event type \mRComplete).
%

%----------------------------------------------------------------------- PFS
\subsubsection{Perfect Forward Secrecy (PFS)}
\label{sec:secrecy}
Informally, PFS captures the idea that session key material remains secret
even if a long-term key leaks in the future.
%
We define PFS for session key material \mSessKey{} as \mPredPfs{} in
Figure~\ref{fig:props}.
%

\begin{figure*}
\begin{align*}
    \mPredPfs \triangleq\ & \forall I, R, \mSessKey, t_2, t_3\mLogicDot
    \mK^{t_3}(\mSessKey)\  \land\ 
    (\mIComplete^{t_2}(I, R, \mSessKey)\, \lor\, \mRComplete^{t_2}(I, R, \mSessKey))
    \rightarrow\\
    &(\exists t_1\mLogicDot \mRevLTK^{t_1}(I) \land t_1 \lessdot t_2)
    \ \lor\ (\exists t_1\mLogicDot \mRevLTK^{t_1}(R) \land t_1 \lessdot t_2)
    \ \lor\ (\exists t_1\mLogicDot \mRevEph^{t_1}(R, \mSessKey))
    \ \lor\ (\exists t_1\mLogicDot \mRevEph^{t_1}(I, \mSessKey))
\end{align*}
\begin{align*}
    \mPredInjI \triangleq\ &
    \forall I, R, \mSessKey, S, t_2\mLogicDot \mIComplete^{t_2}(I, R, \mSessKey, S)
    \rightarrow\\
    &(\exists t_1\mLogicDot \mRStart^{t_1}(R, \mSessKey, S) \land t_1 \lessdot t_2)
    \land (\forall I' R' t_1' \mLogicDot \mIComplete^{t_1'}(I' , R', \mSessKey, S)
        \rightarrow t_1' \doteq t_1)
    \ \ \lor\ \ (\exists t_1\mLogicDot \mRevLTK^{t_1}(R) \land t_1 \lessdot t_2)
\end{align*}
\begin{align*}
    \mPredInjR \triangleq\ &
    \forall I, R, \mSessKey, S, t_2\mLogicDot \mRComplete^{t_2}(I, R, \mSessKey, S)
    \rightarrow\\
    &(\exists t_1\mLogicDot \mIStart^{t_1}(I, R, \mSessKey, S) \land t_1 \lessdot t_2)
    \land (\forall I' R' t_1' \mLogicDot \mRComplete^{t_1'}(I' , R', \mSessKey, S)
        \rightarrow t_1' \doteq t_1)
    \ \ \lor\ \ (\exists t_1\mLogicDot \mRevLTK^{t_1}(I) \land t_1 \lessdot t_2).
\end{align*}
\begin{align*}
    \mPredImpI \triangleq\ &
    \forall I, R, \mSessKey, S, t_1\mLogicDot \mIComplete^{t_1}(I, R, \mSessKey, S)
    \rightarrow\\
      &(\forall I', R', S', t_2\mLogicDot \mRComplete^{t_2}(I', R', \mSessKey, S') \rightarrow
             (I=I' \land R=R' \land S=S'))\\
      &\land (\forall I', R', S', t_1'\mLogicDot
        (\mIComplete^{t_1'}(I', R', \mSessKey, S') \rightarrow t_1' \doteq t_1
        )\\
      &\ \ \ \lor(\exists t_0\mLogicDot \mRevLTK^{t_0}(R) \land t_0 \lessdot t_1)
    \lor(\exists t_0\mLogicDot \mRevEph^{t_0}(R, \mSessKey))
    \lor(\exists t_0\mLogicDot \mRevEph^{t_0}(I, \mSessKey)).
\end{align*}
%
\caption{Formalization of security properties and adversary model.}
\label{fig:props}
\end{figure*}

The first parameter $I$ of the \mIComplete{} event represents the
initiator's identity,
and the second, $R$, represents that $I$ believes $R$ to be playing
the responder role.
%
The third parameter, \mSessKey{}, is the established session key material.
%
The parameters of the \mRComplete{} event are defined analogously.
%
Specifically, the first parameter of \mRComplete{} represents the identity of
whom $R$ believes is playing the initiator role.
%
The essence of the definition is that an adversary only knows \mSessKey{} if they
compromised one of the
parties long-term keys before that party completed the run, or if the adversary
compromised any of the ephemeral keys at any time after a party starts
its protocol run.
%
One way the definition slightly differs from the corresponding \mTamarin{} lemma
is that \mTamarin{} does not allow a disjunction on the left-hand side of an
implication in a universally quantified formula.
%
In the lemma, therefore, instead of the disjunction
$\mIComplete^{t_2}(I, R, \mSessKey)\, \lor\,  \mRComplete^{t_2}(I, R, \mSessKey)$,
we use a single action parametrized by $I$, $R$, and \mSessKey{} to signify that
\emph{either} party has completed their role.
%

%-------------------------------------------------------------------- InjAgree
\subsubsection{Authentication}
\label{sec:authenticationDef}
We prove two different flavors of authentication, the first being classical
\emph{injective agreement} following Lowe~\cite{DBLP:conf/csfw/Lowe97a}, and
the second being an implicit agreement property.
%
Informally, injective agreement guarantees to an initiator $I$ that whenever
$I$ completes a run ostensibly with a responder $R$,
then $R$ has been engaged in the protocol as a responder,
and this run of $I$ corresponds to a unique run of $R$.
%
In addition, the property guarantees to $I$ that the two parties agree on a set
$S$ of parameters associated with the run, including, in particular, the
session key material \mSessKey{}.
%
However, we will treat \mSessKey{} separately for clarity.
%
On completion, $I$ knows that $R$ has access to the session key material.
%
The corresponding property for $R$ is analogous.
%

Traditionally, the event types used to describe injective agreement are called
\emph{Running} and \emph{Commit}, but to harmonize the presentations of
authentication and PFS in this section, we refer to these event types as
\mIStart{} and \mIComplete{} respectively for the initiator, and
\mRStart{} and \mRComplete{} for the responder.
%
For the initiator role we define injective agreement by
\mPredInjI{} in Figure~\ref{fig:props}.

%
The property captures that for an initiator $I$, either the injective agreement
property as described above holds, or the long-term key of the believed
responder $R$ has been compromised before $I$ completed its role.
%
Had the adversary compromised $R$'s long-term key, they could have generated a
message of their liking (different to what $R$ agreed on), and signed this or
computed a $\mathit{MAC}_R$ based on \mPubE{I}, \mPriv{R} and their own chosen
ephemeral key pair $\langle\mPrivE{R},\ \mPubE{R}\rangle$.
%
This places no restrictions on the ephemeral
key reveals, or on the reveal of the initiator's long-term key.
%
For the responder we define the property \mPredInjR{} as in
Figure~\ref{fig:props}.
%

%------------------------------------------------------------- Implicit auth
Unlike PFS, not all \mEdhoc{} methods enjoy the injective agreement property.
%
Hence, we show for all methods a form of \emph{implicit agreement} on all the
parameters mentioned above.
%
We take inspiration from the computational model definitions of implicit
authentication, proposed by Guilhem~et~al.~\cite{DBLP:conf/csfw/GuilhemFW20}, to
modify classical injective agreement into an implicit property.
%
A small but important difference between our definition and theirs, is that
they focus on
authenticating a key and related identities, whereas we extend the more general
concept of agreeing on a set of parameters, starting from the idea of injective
agreement~\cite{DBLP:conf/csfw/Lowe97a}.
%
We use the term \emph{implicit} in this context to denote that a party $A$
assumes that any other party $B$ who knows the session key material \mSessKey{} must
be the intended party, and that $B$ (if honest) will also agree on a set
$S$ of parameters computed by the protocol, one of which is \mSessKey{}.
%
When implicit agreement holds for both roles, upon completion, $A$ is guaranteed
that $A$ has been or is engaged in exactly one protocol run with $B$ in the
opposite role, and that $B$ has been or will be able to agree on $S$.
%
The main difference to injective agreement is that $A$ concludes that if
$A$ sends the last message and this reaches $B$, then $A$ and $B$ have agreed
on $I$, $R$ and $S$.
%
While almost full explicit key authentication, as defined by
Guilhem~et~al.~\cite{DBLP:conf/csfw/GuilhemFW20}, is a similar property, our
definition does not require key confirmation, so our definition is closer to
their definition of implicit authentication.
%
In the \mTamarin{} model we split the property into one lemma for
$I$ (\mPredImpI{}) and one for $R$ (\mPredImpR{}) to save memory during
verification.
%
We show only the definition for $I$ in Figure~\ref{fig:props}, because it is
symmetric to the one for $R$.
%

For implicit agreement to hold for the initiator $I$, the ephemeral keys
can never be revealed.
%
Intuitively, the reason for this is that the implicit agreement relies on that
whomever knows the session key material is the intended responder.
%
An adversary with access to the ephemeral keys and the public keys of
both parties can compute the session key material produced by all methods.
%
However, the responder $R$'s long-term key can be revealed after $I$ completes
its run, because the adversary is still unable to compute $P_e$.
%
The initiator's long-term key can also be revealed at any time without affecting
$I$'s guarantee for the same reason.
%

%------------------------------------------------------- Agreed parameters
\subsubsection{Agreed Parameters}
\label{sec:agreedParams}
The initiator $I$ gets injective and implicit agreement guarantees on the
following partial set $S_P$ of parameters:
\begin{itemize}
    \item the roles played by itself and its peer,
    \item responder identity,
    \item session key material (which varies depending on \mEdhoc{} method),
    \item context identifiers \mCi{} and \mCr{}, and
    \item cipher suites \mSuites{}.
\end{itemize}
%
Because \mEdhoc{} aims to provide identity protection for $I$, there is no
injective agreement guarantee for $I$ that $R$ agrees on the initiator's
identity.
%
For the same reason, there is no such guarantee for $I$ with respect to
the $P_I$ part of the session key material when $I$ uses the \mStat{}
authentication method.
%
There is, however, an implicit agreement guarantee for $I$ that $R$ agrees on
$I$'s identity and the full session key material.
%
Since $R$ completes after $I$, $R$ can get injective agreement guarantees on
more parameters, namely also the initiator's identity and the full session key
material for all methods.
%
The full set of agreed parameters $S_F$ is $S_P \cup \{I, P_I\}$
when $P_I$ is
part of the session key material, and $S_P \cup \{I\}$ otherwise.
%

%------------------------------------------------------- Implied properties
\subsubsection{Inferred Properties}
From the above, other properties can be inferred to hold in our adversary model.
%
Protocols where a party does not get confirmation that their peer knows the
session key material may be susceptible to
\emph{Key-Compromise Impersonation (KCI)}
attacks~\cite{DBLP:conf/ima/Blake-WilsonJM97}.
%
Attacks in this class allow an adversary in possession of a party $A$'s secret
long-term key to coerce $A$ to complete a
protocol run believing it authenticated a certain peer $B$, but where $B$ did
not engage with $A$ at all in a run.
%
Because both our notions of agreement above ensure agreement on identities,
roles and session key material, all methods passing verification of those are
also resistant to KCI attacks.
%

If a party $A$ can be coerced into believing it completed a run with $B$, but
where the session key material is actually shared with $C$ instead, the 
protocol is vulnerable to an \emph{Unknown Key-Share (UKS)}
attack~\cite{DBLP:conf/ima/Blake-WilsonJM97}.
%
For the same reason as for KCI, any method for which our agreement
properties hold is also resistant to UKS attacks.
%

From the injective agreement properties it follows that each party is assured
the identity of its peer upon completion.
%
Therefore, the agreement properties also capture \emph{entity authentication}.
%

%---------------------------------------------------------------------------
\subsection{\mTamarin{}}
\label{sec:tamarin}
We chose \mTamarin{} to model and verify \mEdhoc{} in the symbolic model.
%
\mTamarin{} is an interactive verification tool in which models are specified
as multi-set rewrite rules that define a transition relation.
%
The elements of the multi-sets are called facts and represent the global system
state.
%
Rules are equipped with event annotations called actions.
%
Sequences of actions make up the execution trace, over which
logic formulas are defined and verified.
%

Multi-set rewrite rules with actions are written\\ $ l \ifarrow[e] r $,
where $l$ and $r$ are multi-sets of facts, and $e$ is a multi-set of actions.
%
Facts and actions are $n$-ary predicates over a term algebra, which defines a
set of function symbols, variables and names.
%
\mTamarin{} checks equality of these terms under an equational theory $E$.
%
For example, one can write $ dec(enc(x,y),y) =_E x $
to denote that symmetric decryption reverses the encryption operation under $E$.
%
The equational theory $E$ is fixed per model, and hence we omit the subscript.
%
\mTamarin{} supports let-bindings and tuples as syntactic sugar to simplify
model definitions.
%
It also provides built-in rules for Dolev-Yao adversaries and for
managing their knowledge.
%
We implement events using actions, and parameters associated with events using
terms of the algebra.
%

\subsubsection{Protocol Rules and Equations}
\mTamarin{} allows users to define new function symbols and equational theories.
rules, which are added to the set of considered rules during verification.
For example, in our model we have a symbol to denote authenticated encryption,
for which \mTamarin{} produces a rule of the following form:
%
\begin{small}
\begin{verbatim}
[!KU(k), !KU(m), !KU(ad), !KU(ai)] --[]->
    [!KU(aeadEncrypt(k, m, ad, ai))]
\end{verbatim}
\end{small}
%
to denote that if the adversary knows a key \mT{k}, a message \mT{m}, the
authenticated data \mT{ad}, and an algorithm \mT{ai}, then they can construct
the encryption, and thus get to know the message
\mT{aeadEncrypt(k, m, ad, ai)}.
%

%-------------------------------------------------------------------------- sub
\subsection{\mTamarin{} Encoding of \mEdhoc{}}
\label{sec:modeling}
%
We model all four methods of \mEdhoc{}, namely
\mSigSig, \mSigStat, \mStatSig{} and \mStatStat.
%
Because the methods share a lot of common structure, we derive
their \mTamarin-models from a single specification written with the aid of the
M4 macro language.
%
To keep the presentation brief, we only present the \mStatSig{} metohod, as it
illustrates the use of two different asymmetric authentication methods
simultaneously.
%
The full \mTamarin{} code for all models can be found at~\cite{edhocTamarinRepo}.
%
Variable names used in the code excerpts here are sometimes shortened compared
to the model itself to fit the paper format.
%

\subsubsection{Primitive Operations}
Our model uses the built-in theories of exclusive-or and DH operations, as
in~\cite{DBLP:conf/csfw/DreierHRS18,DBLP:conf/csfw/SchmidtMCB12}.
%
Hashing is modeled via the built-in hashing function symbol augmented
with a public constant as additional input, modelling different
hash functions.
%
The HKDF interface is represented by \mT{expa} for the
expansion operation and \mT{extr} for the extraction operation.
%
Signatures use \mTamarin's built-in theory for \mT{sign} and \mT{verify}
operations.
%
For \mAead{} operations on key \mT{k}, message \mbox{\mT{m}}, additional data \mT{ad}
and algorithm identifier \mT{ai}, we use \mT{aeadEncrypt(m, k, ad, ai)}
for encryption.
%
Decryption with verification of the integrity is defined via the equation
\begin{small}\begin{verbatim}
aeadDecrypt(aeadEncrypt(m, k, ad, ai),
    k, ad, ai) = m.
\end{verbatim}\end{small}
%
The integrity protection of AEAD covers \mT{ad}, and this equation hence requires
an adversary to know \mT{ad} even if they only wish to decrypt the data.
%
To enable the adversary to decrypt without needing to verify the integrity
we add the equation
\begin{small}\begin{verbatim}
decrypt(aeadEncrypt(m, k, ad, ai), k, ai) = m.
\end{verbatim}\end{small}
%
The latter equation is not used by honest parties.
%

\subsubsection{Protocol Environment and Adversary Model}
We model the binding between a party's identity and their long-term
key pairs using the following rules.
%
\begin{small}
\begin{verbatim}
rule registerLTK_SIG:
    [Fr(~ltk)] --[UniqLTK($A, ~ltk)]->
        [!LTK_SIG($A, ~ltk),
         !PK_SIG($A, pk(~ltk)),
         Out(<$A, pk(~ltk)>)]
rule registerLTK_STAT:
    [Fr(~ltk)] --[UniqLTK($A, ~ltk)]->
        [!LTK_STAT($A, ~ltk),
         !PK_STAT($A, 'g'^~ltk),
         Out(<$A, 'g'^~ltk>)]
\end{verbatim}
\end{small}
%
The two rules register long-term
key pairs for the \mSig{}- and \mStat{}-based methods respectively.
%
The fact \verb|Fr(~ltk)| creates a fresh term \mT{ltk}, representing a long-term
secret key, not known to the adversary.
%
The fact \verb|Out(<$A, pk(~ltk)>)| sends the identity of the party
owning the long-term key and the corresponding public key to the adversary.
%
The event \mT{UniqLTK} together with a corresponding restriction models the fact
that the each party is associated with exactly one long-term key.
%
Consequently, an adversary cannot register additional long-term keys for an
identity.
%
In line with the \mEdhoc{} \mSpec{}, this models an external mechanism
ensuring that long term keys are bound to correct identities, e.g.,
a certificate authority.
%

We rely on \mTamarin's{} built-in message deduction rules for a Dolev-Yao adversary.
%
To model an adversary compromising long-term keys, i.e., events of type
\mRevLTK{}, and revealing ephemeral keys, i.e., events of type
\mRevEph{}, we use standard reveal rules.
%
The timing of reveals as modelled by these events is important.
%
The long-term keys can be revealed on registration, before protocol execution.
%
The ephemeral key of a party can be revealed when the party completes,
i.e., at events of type \mIComplete{} and \mRComplete.~\footnote{A stronger, and perhaps more realistic, model would reveal ephemeral keys upon
creation at the start of the run, but we failed to get \mTamarin{} to
terminate on this.}
%

\subsubsection{Protocol Roles}
We model each method of the protocol with four rules: \mT{I1}, \mT{R2}, \mT{I3}
and \mT{R4} (with the method suffixed to the rule name).
%
Each of these represent one step of the protocol as run by the initiator $I$
or the responder $R$.
%
The rules correspond to the event types \mIStart, \mRStart, \mIComplete,  and
\mRComplete, respectively.
%
Facts prefixed with \mT{StI} carry state information between \mT{I1} and \mT{I3}.
%
A term unique to the current thread, \mT{tid}, links two rules to a given state fact.
%
Similarly, facts prefixed with \mT{StR} carry state information between the
responder role's rules.
%
Line 28 in the \mT{R2\_STAT\_SIG} rule shown below illustrate one such use of state
facts.
%

We do not model the error message that $R$ can send in response to message
\mMsgone, and hence our model does not
capture the possibility for $R$ to reject $I$'s offer.
%

We model the XOR encryption of \mT{CIPHERTEXT\_2} with the key \mT{K\_2e} using
\mTamarin{}'s built in theory for XOR, and allow each term of the encrypted
element to be attacked individually.
%
That is, we first expand \mT{K\_2e} to as many key-stream terms as there are
terms in the plaintext tuple by applying the \mHkdfExpand{} function to unique
inputs per term.
%
We then XOR each term in the plaintext with its own key-stream term.
%
This models the \mSpec{} closer than if we would have XORed \mT{K\_2e} as a
single term onto the plaintext tuple.
%
The XOR encryption can be seen in lines 19--22 in the listing of
\mT{R2\_STAT\_SIG} below.
%
\begin{small}
\begin{verbatim}
1 rule R2_STAT_SIG:
2 let
3    agreed = <CS0, CI, ~CR>
4    gx = 'g'^xx
5    data_2 = <'g'^~yy, CI, ~CR>
6    m1 = <'STAT', 'SIG', CS0, CI, gx>
7    TH_2 = h(<$H0, m1, data_2>)
8    prk_2e = extr('e', gx^~yy)
9    prk_3e2m = prk_2e
10   K_2m = expa(<$cAEAD0, TH_2, 'K_2m'>,
11               prk_3e2m)
12   protected2 = $V // ID_CRED_V
13   CRED_V = pkV
14   extAad2 = <TH_2, CRED_V>
15   assocData2 = <protected2, extAad2>
16   MAC_2 = aead('e', K_2m, assocData2,
                  $cAEAD0)
17   authV = sign(<assocData2, MAC_2>, ~ltk)
18   plainText2 = <$V, authV>
19   K_2e = expa(<$cAEAD0, TH_2,
                 'K_2e'>, prk_2e)
20   K_2e_1 = expa(<$cAEAD0, TH_2,
                   'K_2e', '1'>, prk_2e)
21   K_2e_2 = expa(<$cAEAD0, TH_2,
                   'K_2e', '2'>, prk_2e)
22   CIPHERTEXT_2 = <$V XOR K_2e_1,
                     authV XOR K_2e_2>
23   m2 = <data_2, CIPHERTEXT_2>
24   exp_sk = <gx^~yy>
25 in
26   [!LTK_SIG($V, ~ltk), !PK_SIG($V, pkV),
      In(m1), Fr(~CR), Fr(~yy), Fr(~tid)]
27   --[ExpRunningR(~tid, $V, exp_sk, agreed),
        R2(~tid, $V, m1, m2)]->
28   [StR2_STAT_SIG($V, ~ltk, ~yy, prk_3e2m,
              TH_2, CIPHERTEXT_2, gx^~yy,
              ~tid, m1, m2, agreed),
29    Out(m2)]
\end{verbatim}
\end{small}
%

To implement events and
to bind them to parameters, we use actions.
%
For example, the action \verb|ExpRunningR(~tid, $V, exp_sk, agreed)| in line 27
above implements binding of an event of type \mRStart{} to the parameters and session key
material.
%

As explained in Section~\ref{sec:agreedParams}, it is not possible to show
injective agreement on session key material when it includes
$P_I$ (not visible in the rule \mT{R2\_STAT\_SIG}).
%
Therefore, we use certain actions to implement events that include $P_I$ in the
session key material and other actions that do not.
%
Session key material which includes (resp. does not include) $P_I$ is referred
to as \mT{imp\_sk} (resp. \mT{exp\_sk}) in the
\mTamarin{} model.
%
In the case of \mSigSig{} and \mSigStat, therefore, \mT{imp\_sk} is the same as
\mT{exp\_sk}.
%

%-------------------------------------------------------------------------- sub
\subsection{\mTamarin{} Encoding of Properties}
\label{sec:propertyFormalization}
The properties and adversary model we defined in
Section~\ref{sec:desired-properties} translate directly into \mTamarin's logic,
using the straightforward mapping of events to the actions emitted from the model.
%
As an example, we show the lemma for verifying the property \mPredPfs.
%
\begin{small}
\begin{verbatim}
1 lemma secrecyPFS:
2 all-traces
3  "All u v sk #t3 #t2.
4   (K(sk)@t3 & CompletedRun(u, v, sk)@t2) ==>
5       ( (Ex #t1. LTKRev(u)@t1 & #t1 < #t2)
6       | (Ex #t1. LTKRev(v)@t1 & #t1 < #t2)
7       | (Ex #t1. EphKeyRev(sk)@t1))"
\end{verbatim}
\end{small}
%
As mentioned earlier, the action \mT{CompletedRun(u, v, sk)} in line 4 is
emitted by both the rules \mT{I3} and \mT{R4}, and corresponds
to the disjunction of events $\mIComplete^{t_2} \lor \mRComplete^{t_2}$ in the
definition of \mPredPfs{} in Section~\ref{sec:secrecy}.
%
Similarly, \mT{EphKeyRev(sk)} in line 7 models that the ephemeral
key is revealed for either $I$ or $R$, or both.
%
